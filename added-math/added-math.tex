% NOTE: comando definito dal package "mathtools"
\DeclarePairedDelimiter{\norm}{\lVert}{\rVert}
\DeclarePairedDelimiter{\abs}{\lvert}{\rvert}
\DeclarePairedDelimiter\Bracket{\lbrack}{\rbrack}

\makeatletter
\def\mathcolor#1#{\@mathcolor{#1}}
\def\@mathcolor#1#2#3{%
	\protect\leavevmode
	\begingroup
		\color#1{#2}#3%
	\endgroup
}
\makeatother

% NOTE: insiemistica
\newcommand{\vuoto}{\emptyset}
\newcommand{\e}{\( \varepsilon \)}
% \def\eps{\varepsilon}

% NOTE: grammatiche
\newcommand{\produce}{\longrightarrow}
\newcommand{\deriva}{\implies}
\newcommand{\derivamultiplo}{\underset{\deriva}{+}}
\newcommand{\derivanumero}[1]{\underset{\deriva}{#1}}

% NOTE: "uq" sta per "upquote"
\newcommand{\uq}{\text{\textquotesingle}}

% NOTE: parentesi matematiche
\newcommand{\ob}{\overbrace}
\newcommand{\ub}{\underbrace}

% NOTE: notazione insiemistica
\newcommand{\N}{\mathbb{N}}	% natural
\newcommand{\Z}{\mathbb{Z}}	% integers
\newcommand{\I}{\mathbb{I}}	% irrational
\newcommand{\Q}{\mathbb{Q}}	% rational
\newcommand{\R}{\mathbb{R}}	% real
\newcommand{\C}{\mathbb{C}}	% complex

% NOTE: simbolo di fine dimostrazione
\renewcommand\qedsymbol{\( \blacksquare \)}

% NOTE: teoremi, lemmi e nota bene
\newtheorem{theorem}{Teorema}
\newtheorem{lemma}[theorem]{Lemma}
\newtheorem{definition}{Definizione}

\newtheorem*{remark}{Ricorda}
\newtheorem*{note}{Nota}
\newtheorem*{observation}{OSS.}
