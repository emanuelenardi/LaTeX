% NOTE: "xcolor" - permette di definire colori personalizzati
\usepackage[
	,dvipsnames
	,svgnames
	,x11names
	,table
]{xcolor}

% NOTE: Definizione nuovi colori
\definecolor{ashgrey}{rgb}{0.7, 0.75, 0.71}
\definecolor{burgundy}{rgb}{0.5, 0.0, 0.13}
\definecolor{cyan}{rgb}{0.0,0.6,0.6}
\definecolor{darkblue}{rgb}{0.0,0.0,0.6}
\definecolor{gray}{rgb}{0.4,0.4,0.4}

% NOTE: TODO
% \colorlet{accentColor}{blue}
% \colorlet{purple}{MediumPurple1}
% \colorlet{greenYellow}{Chartreuse2}

% NOTE: Abbreviazioni colori
\newcommand{\darkblue}[1]{\textcolor{darkblue}{#1}}
\newcommand{\cyan}[1]{\textcolor{cyan}{#1}}
\newcommand{\gray}[1]{\textcolor{gray}{#1}}

% NOTE: "pifont" - introduce i simboli \cmark e \xmark
\usepackage{pifont}

% NOTE: Marking
\newcommand{\cmark}{\textcolor{ForestGreen}{\ding{51}}}
\newcommand{\xmark}{\textcolor{Red}{\ding{55}}}

\newcommand{\tick}{\cmark}

% OPTIMIZE: rimuovere dai documenti
% \newcommand{\blue}[1]{\textcolor{blue}{#1}}
% \newcommand{\azure}[1]{\textcolor{SkyBlue}{#1}}
% \newcommand{\green}[1]{\textcolor{ForestGreen}{#1}}
% \newcommand{\greenlight}[1]{\textcolor{green!30}{#1}}
% \newcommand{\red}[1]{\textcolor{red}{#1}}
% \newcommand{\orange}[1]{\textcolor{red!50}{#1}}
% \newcommand{\purple}[1]{\textcolor{purple}{#1}}
