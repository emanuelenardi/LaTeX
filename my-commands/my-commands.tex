% NOTE: Permette di inserisce una pagina bianca
\newcommand{\blankpage}{%
	\null%
	\thispagestyle{empty}%
	\addtocounter{page}{-1}%
	\newpage%
}

% NOTE: Simbolo per riferimento a materiali esterni
\newcommand{\ExternalLink}{
	\tikz[x = 1.2ex, y = 1.2ex, baseline = -0.05ex]{
		\begin{scope}[x = 1ex, y = 1ex]
			\clip (-0.1,-0.1) --++ (-0, 1.2) --++ (0.6, 0) --++ (0, -0.6) --++ (0.6, 0) --++ (0, -1);
			\path[draw, line width = 0.5, rounded corners = 0.5] (0,0) rectangle (1,1);
		\end{scope}
		\path[draw, line width = 0.5] (0.5, 0.5) -- (1, 1);
		\path[draw, line width = 0.5] (0.6, 1) -- (1, 1) -- (1, 0.6);
	}
}

\newcommand{\mail}[1]{\href{mailto:#1}{\texttt{#1}\ExternalLink}}

\newcommand{\mysection}[2]{\section[#1]{#1\\[.5ex]\normalsize\textit{#2}}}
\newcommand{\mysubsection}[2]{\subsection[#1]{#1\\[.5ex]\normalsize\textit{#2}}}

\newcommand{\omissis}{[\textellipsis\unkern]}

% NOTE: Linea spessa sul testo sottostante
\newcommand{\soutthick}[1]{
	\renewcommand{\ULthickness}{1.0pt}	% 2.4
		\sout{#1}%
	\renewcommand{\ULthickness}{.4pt}	% Resetting to ulem default
}
