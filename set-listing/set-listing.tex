% NOTE: "listingsutf8" - permette di evitare problemi di codifica NEI FILE CARICATI ESTERNAMENTE
\usepackage{listingsutf8}
\usepackage{alltt}

% NOTE: Abbreviazioni inserimento codice
\newcommand\code[1]{\texttt{#1}}
\newcommand\html[1]{\lstinline[style = HTML]|#1|}
\newcommand\cc[1]{\lstinline[style = C]|#1|}
\newcommand\cpp[1]{\lstinline[style = [11]C++]|#1|}
\newcommand\gradle[1]{\lstinline[style = Gradle]|#1|}
\newcommand\java[1]{\lstinline[style = Java]|#1|}
\newcommand\javascript[1]{\lstinline[style = Javascript]|#1|}
\newcommand\jsp[1]{\lstinline[style = JSP]|#1|}
\newcommand\sml[1]{\lstinline[style = SML]|#1|}
\newcommand\sql[1]{\lstinline[style = SQL]|#1|}
\newcommand\xml[1]{\lstinline[style = XML]|#1|}

% NOTE: Rinomina la didascali
% Listing -> Codice
\renewcommand{\lstlistingname}{Codice}

% NOTE: Rinomina l'indice
% List of Listings -> Lista dei Codici
\renewcommand{\lstlistlistingname}{Lista dei Codici}

\lstloadlanguages{
	[11]C++,
	Java,
	HTML,
	ML,
	SQL,
	XML,
}

\lstset{
	inputencoding = utf8/latin1,	% codifica UTF-8
	extendedchars = true,			% allows extended characters in listings, that means (national) characters of codes 128–255.
	upquote = true,					% determines whether the left and right quote are printed ‘’ or `'. WARNING: This key  requires the textcomp package if true.
	texcl,							% Permette di mostrare formule matematiche nei commenti
	basicstyle = \footnotesize\ttfamily,
	numbers = none,					% Non mostra i numeri lateralmente
	breakatwhitespace = false,		% Non permette di spezzare il codice dove c'è uno spazio
	breaklines = true,				% Permette di spezzare il codice
	columns = fixed,				% incolonnamento corretto dei caratteri
	frame = H,						% Posizionamento corretto dal codice
	prebreak = \raisebox{0ex}[0ex][0ex]{\ensuremath{\rhookswarrow}},
	postbreak = \raisebox{0ex}[0ex][0ex]{\ensuremath{\rcurvearrowse}\space},
	escapeinside = {(*@}{@*)},		% %*codice latex*) CHANGED: (* *) -> (*@ "codice latex" @*)
	tabsize = 4,					% imposta la larghezza del tab pari a 4 spazi
	showspaces = false,				% non mostra gli spazi come caratteri
	showstringspaces = false,		% non mostra gli spazi nelle stringhe come caratteri
	showtabs = false,				% non mostra i tab come carattere
	frame = lines,					% aggiunge una linea sopra ed una linea sotto
	captionpos = b,					% Imposta la posizione della didascalia sul fondo
	aboveskip = 3mm,				% spazio superiore di 0.3 cm
	belowskip = 3mm,				% spazio inferiore di 0.3 cm
	% By default, listings does not support multi-byte encoding for source code. The extendedchar option only works for 8-bits encodings such as latin1. To handle UTF-8, you should tell listings how to interpret the special characters by defining them like so
	literate =
		{á}{{\'a}}1 {é}{{\'e}}1 {í}{{\'i}}1 {ó}{{\'o}}1 {ú}{{\'u}}1
		{Á}{{\'A}}1 {É}{{\'E}}1 {Í}{{\'I}}1 {Ó}{{\'O}}1 {Ú}{{\'U}}1
		{à}{{\`a}}1 {è}{{\`e}}1 {ì}{{\`i}}1 {ò}{{\`o}}1 {ù}{{\`u}}1
		{À}{{\`A}}1 {È}{{\'E}}1 {Ì}{{\`I}}1 {Ò}{{\`O}}1 {Ù}{{\`U}}1
		{ä}{{\"a}}1 {ë}{{\"e}}1 {ï}{{\"i}}1 {ö}{{\"o}}1 {ü}{{\"u}}1
		{Ä}{{\"A}}1 {Ë}{{\"E}}1 {Ï}{{\"I}}1 {Ö}{{\"O}}1 {Ü}{{\"U}}1
		{â}{{\^a}}1 {ê}{{\^e}}1 {î}{{\^i}}1 {ô}{{\^o}}1 {û}{{\^u}}1
		{Â}{{\^A}}1 {Ê}{{\^E}}1 {Î}{{\^I}}1 {Ô}{{\^O}}1 {Û}{{\^U}}1
		{œ}{{\oe}}1 {Œ}{{\OE}}1 {æ}{{\ae}}1 {Æ}{{\AE}}1 {ß}{{\ss}}1
		{ű}{{\H{u}}}1 {Ű}{{\H{U}}}1 {ő}{{\H{o}}}1 {Ő}{{\H{O}}}1
		{ç}{{\c c}}1 {Ç}{{\c C}}1 {ø}{{\o}}1 {å}{{\r a}}1 {Å}{{\r A}}1
		{€}{{\euro}}1 {£}{{\pounds}}1 {«}{{\guillemotleft}}1
		{»}{{\guillemotright}}1 {ñ}{{\~n}}1 {Ñ}{{\~N}}1 {¿}{{?`}}1
		% {$}{\$}1 % TODO: Da testare
	% Another possibility is to replace \usepackage{listings} (in the preamble) with \usepackage{listingsutf8}, but this will only work for \lstinputlisting{...}
}

% NOTE: Rinomina la didascalia dei codici
% NOTE: Rinomina l'indice dei codici
% Listing -> Codice
% List of Listings -> Lista dei Codici
\renewcommand{\lstlistingname}{Codice}
\renewcommand{\lstlistlistingname}{Lista dei Codici}

\lstdefinestyle{Gradle} {
	moredelim = *[s][\color{gray}]{'}{'},
	moredelim = *[s][\color{ForestGreen}]{"}{"},
	emphstyle = \bfseries,
	emph = {
		apply,
		% android,
		compileSdkVersion,
		buildToolsVersion,
		defaultConfig,
		minSdkVersion,
		targetSdkVersion,
		buildTypes,
		release,
		proguardFiles,
		getDefaultProguardFile,
		% dependencies,
		implementation,
		fileTree, testImplementation,
		androidTestImplementation
	},
	emphstyle = {[2]\bfseries\color{violet}},
	emph = {[2]
		applicationId,
		versionCode,
		versionName,
		testInstrumentationRunner,
		minifyEnabled
	},
	emphstyle = {[3]\bfseries\color{orange}},
	emph = {[3]
		true,
		false
	},
}

\lstdefinestyle{HTML} {
	language = HTML,
	morekeywords = {
		placeholder,
		pattern,
		autofocus,
		required,
	},
	moredelim = **[is][{\btHL[fill=green!30]}]{+++}{+++}
}

\lstdefinestyle{Java} {
	language = Java,
	commentstyle = \color{ashgrey},
	emphstyle = {\color{darkgray}},
	emphstyle = {[2]\bfseries},
	keywordstyle = \color{black}\bfseries,
	morecomment = [l][\color{blue}]{//+},		% single line blue comments
	morecomment = [l][\color{red}]{//-},		% single line red comments
	morecomment = [s][\color{blue}]{/*+}{*/},	% multiple line blue comments
	morecomment = [s][\color{red}]{/*-}{*/},	% multiple line red comments
	moredelim = [s][\color{ashgrey}]{/**}{*/},		% multiple line gray comments
	moredelim = [is][\soutthick]{|}{|},						% cancellato
	moredelim = **[is][\color{blue}]{^}{^},					% in blue cumulativo
	moredelim = **[is][\color{ForestGreen}]{***}{***},		% in verde cumulativo
	moredelim = **[is][\color{red}]{~}{~},					% [Alt + 126] in rosso cumulativo
	moredelim = **[is][\btHL]{---}{---},					% rimosso cumulativo
	moredelim = **[is][{\btHL[fill=blue!60]}]{Ž}{Ž},		% [Alt + 0142]  evidenzia cumulativo
	moredelim = **[is][{\btHL[fill=SkyBlue]}]{•}{•},		% [Alt + 0149]  evidenzia cumulativo
	moredelim = **[is][{\btHL[fill=orange!60]}]{Š}{Š},		% [Alt + 0138]  evidenzia cumulativo
	moredelim = **[is][{\btHL[fill=yellow!60]}]{‡}{‡},		% [Alt + 0135]  modificato cumulativo
	moredelim = **[is][{\btHL[fill=ForestGreen]}]{†}{†},	% [Alt + 0134]  evidenzia cumulativo
	moredelim = **[is][{\btHL[fill=green!30]}]{+++}{+++},	%  aggiunto cumulativo
	emph = {
		@Override,
		@param
	},
	emph = {[2]
		Note
	},
	morekeywords = {
		% Java Standard
		PrintWriter, BufferedReader, Integer,
		% Added for Android highlighting
		Intent, LayoutInflater, Menu, MenuInflater, String, Bundle, ViewHolder,
		FragmentManager, FragmentTransaction,
		Object, Object...,
		ArrayList, List, Bitmap, DataSetObserver,
		View, ViewGroup,
		CharSequence, Parcelable, Serializable,
		TextView, ImageView,
		ImageButton, EditText, ListView, ContextMenu,
		LinearLayout, RelativeLayout, TableLayout, FrameLayout,
		IBinder, Runnable,
		% SQLite
		SQLiteDatabase, MySQLiteHelper, Cursor, ArrayAdapter, ContentValues,
		Person, Messenger,
		% Socket Programming
		Socket, ServerSocket,
		InetAddress, DataOutputStream, PrintStream, OutputStream, InputStreamReader, StringTokenizer, FileInputStream,
		% Java Servlet Objects
		HttpServletRequest, HttpServletResponse,
		DateFormat, Integer,
	}
}

\lstdefinestyle{Javascript} {
	language = Javascript,
	commentstyle = \color{ashgrey},
	keywordstyle = \color{black}\bfseries,
}

\lstdefinestyle{JSP} {
	language = JSP,
	commentstyle = \color{ashgrey},
	keywordstyle = \color{black}\bfseries,
	moredelim = [s][\color{ashgrey}]{/**}{*/},		% multiple line gray comments
	moredelim = **[is][\color{red}]{~}{~},			% in rosso cumulativo
}

\lstdefinestyle{Kotlin} {
	language = Kotlin,
	commentstyle = \color{gray}\ttfamily,
	emphstyle = {\color{OrangeRed}},
	identifierstyle = \color{black},
	keywordstyle = \color{NavyBlue}\bfseries,
	% keywordstyle = \color{BurntOrange}\bfseries,
	stringstyle = \color{ForestGreen}\ttfamily,
}

\definecolor{ashgrey}{rgb}{0.7, 0.75, 0.71}
\lstdefinestyle{SML} {
	language = SML,
	commentstyle = \color{ashgrey},
	keywordstyle = \color{black}\bfseries,
	morekeywords = {
		val, type, datatype,
		fn, rec, fun,
		let, local, in, end,
	},
	moredelim = **[is][{\btHL[fill=orange!60]}]{Š}{Š},
	moredelim = **[is][{\btHL[fill=yellow!60]}]{‡}{‡},
	moredelim = **[is][{\btHL[fill=ForestGreen]}]{†}{†},
	moredelim = **[is][{\btHL[fill=green!30]}]{+++}{+++},
	moredelim = **[is][\btHL]{---}{---},
}

\lstdefinestyle{SQL} {
	language = SQL
}

\definecolor{ashgrey}{rgb}{0.7, 0.75, 0.71}
\definecolor{darkblue}{rgb}{0.0,0.0,0.6}
\lstdefinestyle{XML} {
	language = XML,
	commentstyle = \color{ashgrey},
	tag = **[s][\color{darkblue}\renewcommand\delimstyle{\color{black}}]<>,
	identifierstyle = \color{darkblue},
	keywordstyle = \color{cyan},
	stringstyle = \color{ForestGreen},
	moredelim = [is][\soutthick]{|}{|},
	% moredelim = [s][\color{ashgrey}]{<!--}{-->},
	moredelim = **[is][\color{red}]{~}{~},
	moredelim = **[is][\btHL]{---}{---},
	moredelim = **[is][{\btHL[fill=green!30]}]{+++}{+++},
	morekeywords = {
		android, application, encoding,
		id, intent-filter,
		layout_width, layout_height,
		name, orientation, text, type, title, xmlns, version,
	}
}

\makeatletter
\def\lst@DelimPrint#1#2{%
	#1%
		\begingroup
			\lst@mode\lst@nomode \lst@modetrue
			#2\delimstyle\lst@XPrintToken%
		\endgroup
		\lst@ResetToken
	\fi}
\makeatother
\newcommand\delimstyle{}

\lstdefinelanguage{XML} {
	% language = XML,
	morestring = [s]{"}{"},
	morestring = [s]{>}{<},
	morecomment = [l]{\#},
	% morecomment = [s]{<?}{?>},
	morecomment = [s]{<!--}{-->},
}

\lstdefinelanguage{Javascript} {
	language = HTML,
}

\lstdefinelanguage{JSP} {
	language = Java,
	alsolanguage = html,
}

\lstdefinelanguage{Kotlin} {
	sensitive = true,
	comment = [l]{//},
	morecomment = [s]{/*}{*/},
	morestring = [b]",
	morestring = [s]{"""*}{*"""},
	emph = {
		println, return@, forEach,
	},
	keywords = {
		package, as, typealias, this, super, val, var, fun, for, null,
		true, false, is, in, throw, return, break, continue, object, if,
		try, else, while, do, when, yield, typeof, yield, typeof, class,
		interface, enum, object, override, public, private, get, set, import, abstract,
	},
	keywords = {[2]
		@Deprecated, Iterable, Int, Integer, Float,
		Double, String, Runnable, dynamic,
	},
}

\lstdefinelanguage{SML} {
	language = ML,
}

