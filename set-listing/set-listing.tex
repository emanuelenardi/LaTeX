% NOTE: Rinomina la didascali
% Listing -> Codice
\renewcommand{\lstlistingname}{Codice}

% NOTE: Rinomina l'indice
% List of Listings -> Lista dei Codici
\renewcommand{\lstlistlistingname}{Lista dei Codici}

\lstloadlanguages{
	[11]C++,
	Java,
	HTML,
	ML,
	SQL,
	XML,
}

\lstset{
	inputencoding = utf8/latin1,	% codifica UTF-8
	extendedchars = true,			% allows extended characters in listings, that means (national) characters of codes 128–255.
	upquote = true,					% determines whether the left and right quote are printed ‘’ or `'. This key  requires the textcomp package if true.
	texcl,							% Permette di mostrare formule matematiche nei commenti
	basicstyle = \footnotesize\ttfamily,
	numbers = none,					% Non mostra i numeri lateralmente
	breakatwhitespace = false,		% Non permette di spezzare il codice dove c'è uno spazio
	breaklines = true,				% Permette di spezzare il codice
	columns = fixed,				% incolonnamento corretto dei caratteri
	frame = H,						% Posizionamento corretto dal codice
	prebreak = \raisebox{0ex}[0ex][0ex]{\ensuremath{\rhookswarrow}},
	postbreak = \raisebox{0ex}[0ex][0ex]{\ensuremath{\rcurvearrowse}\space},
	escapeinside = {(*@}{@*)},		% %*codice latex*) CHANGED: (* *) -> (*@ "codice latex" @*)
	tabsize = 4,					% imposta la larghezza del tab pari a 4 spazi
	showspaces = false,				% non mostra gli spazi come caratteri
	showstringspaces = false,		% non mostra gli spazi nelle stringhe come caratteri
	showtabs = false,				% non mostra i tab come carattere
	frame = lines,					% aggiunge una linea sopra ed una linea sotto
	captionpos = b,					% Imposta la posizione della didascalia sul fondo
	aboveskip = 3mm,				% spazio superiore di 0.3 cm
	belowskip = 3mm,				% spazio inferiore di 0.3 cm
	% By default, listings does not support multi-byte encoding for source code. The extendedchar option only works for 8-bits encodings such as latin1. To handle UTF-8, you should tell listings how to interpret the special characters by defining them like so
	literate =
		{á}{{\'a}}1 {é}{{\'e}}1 {í}{{\'i}}1 {ó}{{\'o}}1 {ú}{{\'u}}1
		{Á}{{\'A}}1 {É}{{\'E}}1 {Í}{{\'I}}1 {Ó}{{\'O}}1 {Ú}{{\'U}}1
		{à}{{\`a}}1 {è}{{\`e}}1 {ì}{{\`i}}1 {ò}{{\`o}}1 {ù}{{\`u}}1
		{À}{{\`A}}1 {È}{{\'E}}1 {Ì}{{\`I}}1 {Ò}{{\`O}}1 {Ù}{{\`U}}1
		{ä}{{\"a}}1 {ë}{{\"e}}1 {ï}{{\"i}}1 {ö}{{\"o}}1 {ü}{{\"u}}1
		{Ä}{{\"A}}1 {Ë}{{\"E}}1 {Ï}{{\"I}}1 {Ö}{{\"O}}1 {Ü}{{\"U}}1
		{â}{{\^a}}1 {ê}{{\^e}}1 {î}{{\^i}}1 {ô}{{\^o}}1 {û}{{\^u}}1
		{Â}{{\^A}}1 {Ê}{{\^E}}1 {Î}{{\^I}}1 {Ô}{{\^O}}1 {Û}{{\^U}}1
		{œ}{{\oe}}1 {Œ}{{\OE}}1 {æ}{{\ae}}1 {Æ}{{\AE}}1 {ß}{{\ss}}1
		{ű}{{\H{u}}}1 {Ű}{{\H{U}}}1 {ő}{{\H{o}}}1 {Ő}{{\H{O}}}1
		{ç}{{\c c}}1 {Ç}{{\c C}}1 {ø}{{\o}}1 {å}{{\r a}}1 {Å}{{\r A}}1
		{€}{{\euro}}1 {£}{{\pounds}}1 {«}{{\guillemotleft}}1
		{»}{{\guillemotright}}1 {ñ}{{\~n}}1 {Ñ}{{\~N}}1 {¿}{{?`}}1
		% {$}{\$}1 % TODO: Da testare
	% Another possibility is to replace \usepackage{listings} (in the preamble) with \usepackage{listingsutf8}, but this will only work for \lstinputlisting{...}
}

% NOTE: Rinomina la didascalia dei codici
% Listing -> Codice
\renewcommand{\lstlistingname}{Codice}

% NOTE: Rinomina l'indice dei codici
% List of Listings -> Lista dei Codici
\renewcommand{\lstlistlistingname}{Lista dei Codici}
