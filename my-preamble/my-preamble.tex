% ---------------------------------------------------------------------------- %
% ---------------------------------------------------------------------------- %
% ---------------------------------------------------------------------------- %
% Welcome to hell!
% If you are editing this, look. I'm sorry.
% I did not expect you to come here.
% What are doing in this barren wasteland anyway?
% Well, I guess if you've made it this far, you've grown to understand my ugly code.
% But since this is (probably) the worst part, feel free to ask me for help:
%
% emanuele.nardi@studenti.unitn.it
%
% Note to future self:
% This message does not apply to you. Enjoy hell.
% ---------------------------------------------------------------------------- %
% ---------------------------------------------------------------------------- %
% ---------------------------------------------------------------------------- %

\PassOptionsToPackage{export}{adjustbox}				% Gestisce la dimensione massima delle immagini
\PassOptionsToPackage{ruled, vlined, italiano}{algorithm2e}
\PassOptionsToPackage{open,openlevel=1}{bookmarks}		% Segnalibri operto fino al 1° livello
\PassOptionsToPackage{labelfont={sf,bf}}{caption}		% Le didascialia in grassetto, senza grazie
\PassOptionsToPackage{autostyle}{csquotes}				% gestione delle virgolette
\PassOptionsToPackage{useregional,showdow}{datetime2}
\PassOptionsToPackage{bottom}{footmisc}					% Note a piè pagina
\PassOptionsToPackage{os=win}{menukeys}					% Notazione windows
\PassOptionsToPackage{parfill}{parskip}					% Imposta la spaziatura fra i paragrafi
\PassOptionsToPackage{colorinlistoftodos,prependcaption,textsize=tiny}{todonotes}
\PassOptionsToPackage{titles,subfigure}{tocloft}		% Evita errori di caricamento dei pacchetti
\PassOptionsToPackage{normalem}{ulem}					% replaces underlining with italics in text emphasized by \emph

% It produces a clashing error with the tocloft package if the TOC is displayed
% \PassOptionsToPackage{colorlinks=true, allcolors=blue}{hyperref}

% ---------------------------------------------------------------------------- %

% CHANGED: Fa attenzione in quale classe stai compilando il documento
\documentclass[
	% draft,						% makes LaTeX indicate hyphenation and justification problems with a small square in the right-hand margin of the problem line so they can be located quickly by a human. It also suppresses the inclusion of images and shows only a frame where they would normally occur.
	usenames,
	dvipsnames,
	x11names,
	% gray,
	% fleqn,						% Typesets displayed formulas left-aligned instead of centered.
	leqno,							% Places the numbering of formulas on the left hand side instead of the right.
	a4paper,						% Defines the paper size.
	11pt,							% Sets the size of the main font in the document. If no option is specified, 10pt is assumed.
	titlepage,						% Specifies whether a new page should be started after the document title or not. The article class does not start a new page by default, while report and book do.
]{article}%{exam}

% ---------------------------------------------------------------------------- %

% The cmap package is intended to make the PDF files generated by pdflatex
% "searchable and copyable" in acrobat reader and other compliant PDF viewers.
\usepackage{cmap}

% \renewcommand*\familydefault{\sfdefault}
% \usepackage{roboto}					% Option 'sfdefault' only if the base font of the document is to be sans serif
\usepackage{avant}						% Lo utilizzo per il testo
\usepackage[scaled=.9]{sourcecodepro}	% Lo utilizzo per il codice

\usepackage[T1]{fontenc}           % Codifica dei font
\usepackage[utf8]{inputenc}        % Lettere accentate da tastiera
\usepackage[italian]{babel}        % Lingua del documento
\usepackage[italian]{varioref}     % Indicazione del numero di pagina per i riferimenti tramite \vref

\setlength{\marginparwidth}{2cm}   % NOTE: Permette di non avere warning dal package "todonotes"

% NOTE: Mantenee l'ordine dei pacchetti è fonadamentale per non rompere la build
\usepackage{
	algorithm2e,
	% abraces,			% angle brackets
	standalone,			%
	adjustbox,			% boxes
	afterpage,			%
	array,				% Permette di creare delle colonne personalizzate
	amsthm,				% teoremi e dimostrazioni
	amsfonts,			% nomi insiemi numerici
	amssymb,			% leqslant & geqslant
	bigstrut,
	bookmark,			% Crea e gestisce i segnalibri
	booktabs,			% Genera filetti professionali per le tabelle
	% braket,				% Permette l'uso di parentesi angolari
	caption,			% Personalizzazione delle didascalie
	% coffee4,			% Macchie di caffé
	csquotes,			% Permette di personalizzare le citazioni
	xcolor,				% Permette di definire colori personalizzati
	datetime2,			% Gestione e stampa delle date in vari formati
	enumitem,			% Permette di personalizzare gli elenchi puntati
	footmisc,			% Permette di aggiungere note a pié pagina CHANGED: position with "fancyhdr to avoid warning"
	float,				% Definisce l'opzione "H" per gli oggetti fluttuanti
	fancyhdr,			% Permette la personalizzazione di testatina e piede, WARNING: commenta quando usi la documentclass "exam"
	forest,				% Permette di disegnare l'albero di cartelle
	geometry,			% Gestisce i margini della pagina
	graphicx,			% Permette di allegare figure
	lipsum,				% Genera testo fittizio
	lastpage,			% Riferimento all'ultima pagina (LastPage)
	listingsutf8,		% Permette di evitare problemi di codifica NEI FILE CARICATI ESTERNAMENTE
	makecell,
	mathtools,			% mathtools = amsmath + other stuff
	microtype,			% Migliora il riempimento delle righe
	menukeys,			% CARICALO DOPO inputenc e xcolor, non usare hyperref con l'opzione colorlinks
	MnSymbol,			% Fornisce le freccie che utilizzo per andare a capo riga nelle liste di codice
	multirow,			% Tabelle con righe multilinea
	% % NOTE: rompeva la build, da reintegrare
	% parskip,			% Imposta la spaziatura fra i paragrafi
	pifont,				% \cmark e \xmark
	siunitx,			% Unità di misura del SI
	subfig,				% Non insieme al pacchetto subcaption
	tabularx,			% Crea tabelle dalla larghezza dinamica
	textcomp,			% Definisce la macro "textquotesingle"
	ltablex,			% Crea tabelle dalla larghezza dinamica su più pagine
	tocloft,			% Gestione dei Table Of Content
	todonotes,			% Permette l'inserimento di annotazioni
	ulem,				% Tipi di sottolineature
	xargs,				% Use more than one optional parameter in a new commands
}

% NOTE: documento
\usepackage{
	pdflscape,			% Permette di mettere in orizzontale il pdf
	rotating,			% Permette di ruotare la tabella
}

% NOTE: disegnare
\usepackage{
	pgfgantt,			% Permette di disegnare i diagrammi di Gantt
	tikz,				% Pacchetto completo per disegnare su LaTeX
	tikz-er2,			% Permette di disegnare diagrammi ER
}

% NOTE: definizione nuovi comandi o ambienti
\usepackage{
	comment,			% Fornisce l'ambiente dei commenti
	chngcntr,			% It defines the \counterwithin command, useful with parts of a document
	etoolbox,			% Costrutto if-then-else più altri strumenti utili
	pgffor,				% foreach
}

% NOTE: UML
\usepackage{
	tikz-uml,			% Permette di disegnare grafici UML
	ifthen,				% Dipendenza del pacchetto "tikz-uml"
	xstring,			% Dipendenza del pacchetto "tikz-uml"
	calc,				% Dipendenza del pacchetto "tikz-uml"
	pgfopts,			% Dipendenza del pacchetto "tikz-uml"
}

% NOTE: Da caricare dopo "tikz-uml"
\usepackage{
	pgf-umlsd			% Permette di disegnare sequence diagram
}

\usetikzlibrary{
	automata,			% Automi a stati finiti
	calc,				% Esegue calcoli matematici
	matrix,				% Permette di disegnare matrici
	intersections,		% Permette di disegnare in 2D
	positioning,		% Fornisce comandi per posizionare i nodi in modo relativo
	shapes.geometric,	% Forme geometriche
	arrows,				% Frecce
}

% WARNING: bisogna caricarlo per ultimo
\usepackage{hyperref}	% Crea i collegamenti ipertestuali

% ---------------------------------------------------------------------------- %
% ---------------------------------------------------------------------------- %
% ---------------------------------------------------------------------------- %
