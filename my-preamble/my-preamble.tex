% ---------------------------------------------------------------------------- %
% ---------------------------------------------------------------------------- %
% ---------------------------------------------------------------------------- %
% Welcome to hell!
%
% If you are editing this, look. I'm sorry.
% I did not expect you to come here.
% What are doing in this barren wasteland anyway?
% Well, I guess if you've made it this far, you've grown to understand my ugly code.
% But since this is (probably) the worst part, feel free to ask me for help:
%
% emanuele.nardi@studenti.unitn.it
%
% Note to future self:
% This message does not apply to you. Enjoy hell.
% ---------------------------------------------------------------------------- %
% ---------------------------------------------------------------------------- %
% ---------------------------------------------------------------------------- %

% IDEA: migliorare la tipografia
% NOTE: activate = {true,nocompatibility} - activate protrusion and expansion
% NOTE: final - enable microtype; use "draft" to disable
% NOTE: tracking = true, kerning=true, spacing=true - activate these techniques
% NOTE: factor = 1100 - add 10% to the protrusion amount (default is 1000)
% NOTE: stretch = 10, shrink = 10 - reduce stretchability/shrinkability (default is 20/20)
\PassOptionsToPackage{
	% activate = {true,nocompatibility},
	activate = {true},
	final,
	% draft,
	tracking = true,
	kerning = true,
	spacing = true,
	factor = 1100,
	stretch = 10,
	shrink = 10
}{microtype}

% NOTE: languages used in the document
\PassOptionsToPackage{
	english,
	main = italian,
}{babel}

% NOTE: gestisce la dimensione massima delle immagini
\PassOptionsToPackage{
	export,
}{adjustbox}

% NOTE: impostazioni della stampa degli algoritmi
\PassOptionsToPackage{
	linesnumbered,
	ruled,
	vlined,
	italiano,
	onelanguage,
}{algorithm2e}

% NOTE: i segnalibri vengono aperti fino al 1° livello
\PassOptionsToPackage{
	open,
	openlevel = 1,
}{bookmarks}

% NOTE: virgolette
\PassOptionsToPackage{
	autostyle,
}{csquotes}

% NOTE: stampa della data
\PassOptionsToPackage{
	useregional,
	showdow,
}{datetime2}

% NOTE: lista in linea
\PassOptionsToPackage{
	inline,
}{enumitem}

% NOTE: note a piè pagina
\PassOptionsToPackage{
	bottom,
}{footmisc}

% NOTE: metodo di stampa dei box
\PassOptionsToPackage{
	framemethod = tikz,
}{mdframed}

% NOTE: gestione dei "to-do"
% OPTIMIZE: da rimuovere, genera lentezza
% \PassOptionsToPackage{
% 	colorinlistoftodos,
% 	prependcaption,
% 	textsize = tiny
% }{todonotes}

% NOTE: evita errori di caricamento dei pacchetti
\PassOptionsToPackage{
	titles,
}{tocloft}

% NOTE: replaces underlining with italics in text emphasized by \emph
\PassOptionsToPackage{
	normalem
}{ulem}

% NOTE: righe colorate
\PassOptionsToPackage{
	dvipsnames,
	x11names,
	table,
}{xcolor}

% FIXME: t produces a clashing error with the tocloft package if the TOC is displayed
\PassOptionsToPackage{
	colorlinks = true,
	% allcolors = blue,
	linkcolor 	= red,
	anchorcolor = black,
	citecolor 	= green,
	filecolor 	= cyan,
	menucolor 	= red,
	runcolor 	= cyan,
	urlcolor 	= magenta,
}{hyperref}

% ---------------------------------------------------------------------------- %

% REVIEW: Fa attenzione in quale classe stai compilando il documento
% NOTE: "draft" makes LaTeX indicate hyphenation and justification problems with a small square in the right-hand margin of the problem line so they can be located quickly by a human. It also suppresses the inclusion of images and shows only a frame where they would normally occur;
% NOTE: "fleqn" typesets displayed formulas left-aligned instead of centered;
% NOTE: "leqno" places the numbering of formulas on the left hand side instead of the right;
% NOTE: "a4paper" defines the paper size;
% NOTE: "11pt" dets the size of the main font in the document. If no option is specified, 10pt is assumed.
% NOTE: specifies whether a new page should be started after the document title or not. The article class does not start a new page by default, while report and book do.
\documentclass[
	% draft,
	% gray,
	% fleqn,
	leqno,
	a4paper,
	11pt,
	titlepage,
]{article}

% ---------------------------------------------------------------------------- %

% NOTE: The cmap package is intended to make the PDF files generated by pdflatex
% "searchable and copyable" in acrobat reader and other compliant PDF viewers.
\usepackage{cmap}
% \usepackage[noTeX]{mmap}       % cmap + mathematics (Unicode)

% IDEA: fonts
% \renewcommand*\familydefault{\sfdefault}
% \usepackage{roboto}
% \usepackage{avant}

% NOTE: introdotto dal progetto calliope
% \usepackage[default, osfigures, scale=0.95]{opensans}
\usepackage[scaled=.8]{sourcecodepro}

% IDEA: standard options of a document
\usepackage[T1]{fontenc}
\usepackage[utf8]{inputenc}
% NOTE: l'ultima dev'essere la lingua prncipale del documento
\usepackage{babel}
\usepackage[italian]{varioref}

% HACK: permette di non avere warning dal package "todonotes"
\setlength{\marginparwidth}{2cm}

% IDEA: compialre indipendente i file
% \usepackage{subfiles}

% IDEA: miscellaneous
% NOTE: "adjustbox" -
% NOTE: "afterpage" -
% NOTE: "bookmark" - crea e gestisce i segnalibri
% NOTE: "csquotes" - permette di personalizzare le citazioni
% NOTE: "datetime2" - gestione e stampa delle date in vari formati
% NOTE: "enumitem" - permette di personalizzare gli elenchi puntati
% NOTE: "fancyhdr" - permette la personalizzazione di testatina e piede WARNING: commenta quando usi la documentclass "exam"
% NOTE: "footmisc" - permette di aggiungere note a pié pagina CHANGED: cambiata posizione con "fancyhdr to avoid warning"
% NOTE: "forest" - permette di disegnare alberi di cartelle
% NOTE: "lastpage" - riferimento all'ultima pagina (LastPage)
% NOTE: "lipsum" - genera testo fittizio
% NOTE: "mdframed" -
% NOTE: "microtype" - migliora il riempimento delle righe
% NOTE: "pifont" - introduce i simboli \cmark e \xmark
% NOTE: "standalone" -
% NOTE: "tocloft" - gestione dei Table Of Content
% NOTE: "ulem" - diversi tipi di sottolineatura
% NOTE: "xcolor" - permette di definire colori personalizzati
\usepackage{
	adjustbox,
	afterpage,
	bookmark,
	csquotes,
	datetime2,
	enumitem,
	fancyhdr,
	footmisc,
	forest,
	lastpage,
	lipsum,
	mdframed,
	microtype,
	pifont,
	standalone,
	tocloft,
	ulem,
}

% IDEA: figure
% NOTE: "graphicx" - inserimento figure nel docuemento
% NOTE: "float" - definisce l'opzione "H" per gli oggetti fluttuanti
% NOTE: "wrapfig" - inserimento di figure di fianco al testo
\usepackage{
	graphicx,
	float,
	subcaption,
	wrapfig,
}

% WARNING: package deprecati
% OPTIMIZE: subfig -> subcaption
% OPTIMIZE: caption -> subcaption

% NOTE: opzionale
% \usepackage{
% 	coffee4,			% Macchie di caffé
% }

% IDEA: algoritmi e codice
% NOTE: "algorithm2e" - specifica degli algoritmi
% NOTE: "algpseudocode" - speudocodice
% NOTE: "alltt" - ridefinisce l'ambiente "verbatim"
% NOTE: "listingsutf8" - permette di evitare problemi di codifica NEI FILE CARICATI ESTERNAMENTE
\usepackage{
	% algorithmic,
	% algorithmicx,
	% algpseudocode,
	algorithm2e,
	alltt,
	listingsutf8,
}

% IDEA: tabelle
% NOTE: "array" - permette di creare delle colonne personalizzate
% NOTE: "bigstrut" -
% NOTE: "booktabs" - genera filetti professionali per le tabelle
% NOTE: "colortbl" - righe e celle colorate
% NOTE: "diagbox" - diagonal rule on a cell
% NOTE: "ltablex" - crea tabelle dalla larghezza dinamica su più pagine
% NOTE: "makecell" -
% NOTE: "multirow" - tabelle con righe multilinea
% NOTE: "tabularx" - crea tabelle dalla larghezza dinamica
\usepackage{
	array,
	tabu,
	xcolor,
	bigstrut,
	booktabs,
	colortbl,
	diagbox,
	ltablex,
	makecell,
	multirow,
	tabularx,
}

% WARNING: Mantenee l'ordine dei pacchetti è fonadamentale per non rompere la build
% IDEA: math suymbols
% NOTE: "amsthm" - teoremi e dimostrazioni
% NOTE: "amsfonts" - nomi insiemi numerici
% NOTE: "amssymb" - leqslant & geqslant
% NOTE: "mathtools" - mathtools = amsmath + other stuff
% NOTE: "MnSymbol" - fornisce le freccie che utilizzo per andare a capo riga nelle liste di codice
% NOTE: "abraces" - angle brakets fine tuning
% NOTE: "braket" - permette l'uso di parentesi angolari
% WARNING: "semantic" - da rimuovere - introduce astrazione
% NOTE: "nicefrac" - divisione in linea
% NOTE: "textgreek" - caratteri greci
% NOTE: "siunitx" - unità di misura del SI
% WARNING: "semantic" da caricare sopo "amsmath"
\usepackage{
	amsthm,
	amsfonts,
	amssymb,
	MnSymbol,
	mathtools,
	% braket,
	semantic,
	nicefrac,
	% textgreek,
	siunitx
}

% IDEA: rotazione di documento e tabelle
% NOTE: "geometry" - gestisce i margini della pagina
% NOTE: "pdfpages" - inserimento di pdf all'interno del documento
% NOTE: "pdflscape" - rotazione pagine del documento
% NOTE: "rotating" - rotazione tabelle
\usepackage{
	geometry,
	% pdfpages,
	pdflscape,
	rotating,
}

% IDEA: annotazioni - note a margine
% TODO: usalo per i commenti nella tesi
\usepackage{
	todonotes,
	marginnote,
	mparhack,
	marginfix,
}

% IDEA: disegnare grafici
% NOTE: permette di disegnare i diagrammi di Gantt
% NOTE: pacchetto completo per disegnare su LaTeX
% NOTE: permette di disegnare diagrammi ER
\usepackage{
	pgfgantt,
	tikz,
	tikz-er2,
}

% IDEA: definizione nuovi comandi o ambienti
% NOTE: "chngcntr" - defines the "\counterwithin" command, useful with parts of a document
% NOTE: "comment" - fornisce l'ambiente dei commenti
% NOTE: "etoolbox" - contiene il costrutto if-then-else più altri strumenti utili
% NOTE: "pgffor" - fornisce il costrutto "foreach"
% NOTE: "textcomp" - definisce la macro "textquotesingle" e formatta i numeri
% NOTE: "xargs" - use more than one optional parameter in a new commands
\usepackage{
	chngcntr,
	comment,
	etoolbox,
	pgffor,
	textcomp,
	xargs,
}

% IDEA: disegnare grafici UML
% NB: Da caricare dopo "tikz-uml"
\usepackage{
	% tikz-uml,
	ifthen,
	xstring,
	calc
}

% IDEA: disegnare sequence diagram
% NB: Da caricare dopo "tikz-uml"
\usepackage{
	% pgf-umlsd,
}

% NOTE: sotto-librerie del pacchetto tikz
\usetikzlibrary[
	% arrows,
	automata,
	% calc,
	% intersections,
	% matrix,
	positioning,
	% shapes.geometric,
	% tikzmark,
	% decorations.text,
	% decorations.pathmorphing
]

% NOTE: optional
% \usetikzlibrary[
% 	external
% ]

% \usepackage{pgfplots}
% \usepgfplotslibrary[
% 	external
% ]

% NOTE: opzioni per la pre-compilazioni delle immagini create con tikz
% \tikzexternalize[
% 	mode = graphics if exists,
% 	figure list = true,
% 	prefix = ./assets/figures-tikz/
% ]

% NB: bisogna caricarlo per ultimo
\usepackage{hyperref}

\newboolean{maindoc}
\setboolean{maindoc}{false}

% ---------------------------------------------------------------------------- %
% ---------------------------------------------------------------------------- %
% ---------------------------------------------------------------------------- %

% TODO: da integrare
% \usepackage[
% 	bibstyle = authoryear,
% 	citestyle = numeric,
% 	natbib = false,
% 	backend = biber,
% 	sorting = nyt,
% 	autocite = plain,
% 	hyperref = true,
% 	backref = true,
% 	dashed = false
% ]{biblatex}
