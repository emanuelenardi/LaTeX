% ---------------------------------------------------------------------------- %
% ---------------------------------------------------------------------------- %
% ---------------------------------------------------------------------------- %
% Welcome to hell!
%
% If you are editing this, look. I'm sorry.
% I did not expect you to come here.
% What are doing in this barren wasteland anyway?
% Well, I guess if you've made it this far, you've grown to understand my ugly code.
% But since this is (probably) the worst part, feel free to ask me for help:
%
% emanuele.nardi@studenti.unitn.it
%
% Note to future self:
% This message does not apply to you. Enjoy hell.
% ---------------------------------------------------------------------------- %
% ---------------------------------------------------------------------------- %
% ---------------------------------------------------------------------------- %

% NOTE: "draft" makes LaTeX indicate hyphenation and justification problems with a small square in the right-hand margin of the problem line so they can be located quickly by a human. It also suppresses the inclusion of images and shows only a frame where they would normally occur;
% NOTE: "fleqn" typesets displayed formulas left-aligned instead of centered;
% NOTE: "leqno" places the numbering of formulas on the left hand side instead of the right;
% NOTE: "a4paper" defines the paper size;
% NOTE: "11pt" dets the size of the main font in the document. If no option is specified, 10pt is assumed.
% NOTE: specifies whether a new page should be started after the document title or not. The article class does not start a new page by default, while report and book do.
\documentclass[
	% ,draft
	% ,gray
	% ,fleqn
	,leqno
	,a4paper
	,10pt
	,titlepage
]{article}

% ---------------------------------------------------------------------------- %
% ---------------------------------------------------------------------------- %

% NOTE: The cmap package is intended to make the PDF files generated by pdflatex
% "searchable and copyable" in acrobat reader and other compliant PDF viewers.
\usepackage{cmap}
% \usepackage[noTeX]{mmap}       % cmap + mathematics (Unicode)

\usepackage{ifxetex}

\ifxetex
	\usepackage[no-math]{fontspec}

	\usepackage{polyglossia}
	\setmainlanguage{italian}

	\usepackage{xltxtra}
	\newcommand{\tex}{\TeX}
	\newcommand{\latex}{\LaTeX}
	\newcommand{\xelatex}{\XeLaTeX}

	% TODO: specifica font predefiniti con xelatex
\else
	\usepackage[T1]{fontenc}
	\usepackage[utf8]{inputenc}
	% NOTE: languages used in the document
	% NB: l'ultima dev'essere la lingua principale del documento
	\usepackage[english, main=italian]{babel}

	% NOTE: introdotto dal progetto calliope
	% \usepackage[default, osfigures, scale=0.95]{opensans}
	\usepackage[scaled=.8]{sourcecodepro}

	% IDEA: "microtype" - migliora il riempimento delle righe
	% NOTE: activate = {true,nocompatibility} - activate protrusion and expansion
	% NOTE: final - enable microtype; use "draft" to disable
	% NOTE: tracking = true, kerning=true, spacing=true - activate these techniques
	% NOTE: factor = 1100 - add 10% to the protrusion amount (default is 1000)
	% NOTE: stretch = 10, shrink = 10 - reduce stretchability/shrinkability (default is 20/20)
	\usepackage[
		% ,activate	= {true,nocompatibility}
		,activate	= {true}
		,final
		% ,draft
		,tracking	= true
		,kerning	= true
		,spacing	= true
		,factor		= 1100
		,stretch	= 10
		,shrink		= 10
	]{microtype}
\fi

% ---------------------------------------------------------------------------- %
% ---------------------------------------------------------------------------- %

% NOTE: Definizione nuovi colori
\definecolor{ashgrey}{rgb}{0.7, 0.75, 0.71}
\definecolor{burgundy}{rgb}{0.5, 0.0, 0.13}
\definecolor{cyan}{rgb}{0.0,0.6,0.6}
\definecolor{darkblue}{rgb}{0.0,0.0,0.6}
\definecolor{gray}{rgb}{0.4,0.4,0.4}

% NOTE: Definizione nomi colori personalizzati
\colorlet{accentColor}{blue}
\colorlet{purple}{MediumPurple1}
\colorlet{greenYellow}{Chartreuse2}

% NOTE: Abbreviazioni colori
\newcommand{\darkblue}[1]{\textcolor{darkblue}{#1}}
\newcommand{\cyan}[1]{\textcolor{cyan}{#1}}
\newcommand{\gray}[1]{\textcolor{gray}{#1}}

% \newcommand{\blue}[1]{\textcolor{blue}{#1}}
% \newcommand{\azure}[1]{\textcolor{SkyBlue}{#1}}
\newcommand{\green}[1]{\textcolor{ForestGreen}{#1}}
% \newcommand{\greenlight}[1]{\textcolor{green!30}{#1}}
\newcommand{\red}[1]{\textcolor{red}{#1}}
% \newcommand{\orange}[1]{\textcolor{red!50}{#1}}
% \newcommand{\purple}[1]{\textcolor{purple}{#1}}

% NOTE: Marking
\newcommand{\cmark}{\textcolor{ForestGreen}{\ding{51}}}
\newcommand{\xmark}{\textcolor{Red}{\ding{55}}}

\newcommand{\tick}{\cmark}
\colorlet{bodyColor}{blue!50!white}
\colorlet{driverColor}{blue}

% NOTE 'datetime2' - gestione e stampa delle date in vari formati
\usepackage[useregional,showdow]{datetime2}

% NOTE: gestisce la dimensione massima delle immagini
\usepackage[export]{adjustbox}

% IDEA: figure
% NOTE: "graphicx" - inserimento figure nel docuemento
% NOTE: "float" - definisce l'opzione "H" per gli oggetti fluttuanti
% NOTE: "wrapfig" - inserimento di figure di fianco al testo
\usepackage{
	graphicx,
	float,
	subcaption,
	wrapfig,
}


% WARNING: package deprecatig
% OPTIMIZE: footmisc -> TODO
% OPTIMIZE: subfig -> subcaption
% OPTIMIZE: caption -> subcaption

% NOTE: "algorithm2e" - specifica degli algoritmi
% NOTE: "algpseudocode" - speudocodice
% NOTE: "alltt" - ridefinisce l'ambiente "verbatim"
\usepackage{
	algorithmic,
	algpseudocode,
}
% NOTE: from https://www.thomasdenney.co.uk/blog/2017/4/18/typesetting-algorithms-with-latex/
\algtext*{EndWhile}
\algtext*{EndFor}
\algtext*{EndIf}
\algtext*{EndFunction}

\algnewcommand{\SIf}[1]{\State\algorithmicif\ #1\ \algorithmicthen}
\algnewcommand{\SElseIf}[1]{\State\algorithmicelse\ \algorithmicif\ #1\ \algorithmicthen}
\algnewcommand{\SElse}{\State\algorithmicelse\ }
\algnewcommand{\SWhile}[1]{\State\algorithmicwhile\ #1\ \algorithmicdo}
\algnewcommand{\SFor}[1]{\State\algorithmicfor\ #1\ \algorithmicdo}
\algnewcommand{\SForAll}[1]{\State\algorithmicforall\ #1\ \algorithmicdo}

% NOTE: Impostazione delle didascalie
\captionsetup{
	figureposition = bottom,	% opzione analoga alla successiva per le figure
	tableposition = top,		% ordina al programma di inserire uno spazio adeguato tra didascalia e tabella
	font = small,				% produce didascalie in corpo più piccolo
	format = hang,				% allinea (hang) alla prima riga quelle successive
	labelfont = {sf,bf}			% imposta l’etichetta della didascalia in caratteri senza grazie, in grassetto
}

\newcolumntype{C}{>{$}c<{$}}
\newcolumntype{L}{>{$}l<{$}}
\newcolumntype{R}{>{$}r<{$}}

% OPTIMIZE: ci sono modi migliori per farlo
\newcolumntype{!}{>{\global\let\currentrowstyle\relax}}
\newcolumntype{^}{>{\currentrowstyle}}
\newcommand{\rowstyle}[1]{\gdef\currentrowstyle{#1}%
	#1\ignorespaces%
}
%
% NOTE: "geometry" - gestisce i margini della pagina
\usepackage{geometry}

% NOTE: "paper" - formato carta A4
% NOTE: "margin" - specifica tutti i margini
% NOTE: "heightrounded" - TODO
\geometry{
	% ,margin	= 2cm
	,top	= 1.5cm
	,bottom	= 2cm
	,left	= 2cm
	,right	= 2cm
	,heightrounded
}

% NOTE 'ulem' - permette di avere diversi tipi di sottolineatura
% NB 'normalem' - replaces underlining with italics in text emphasized by \emph
\usepackage[normalem]{ulem}

% NOTE Linea spessa sul testo sottostante
\newcommand{\soutthick}[1]{
	\renewcommand{\ULthickness}{1.0pt}	% 2.4
		\sout{#1}%
	\renewcommand{\ULthickness}{.4pt}	% Resetting to ulem default
}

% NOTE 'mdframed' - TODO
\usepackage[framemethod = tikz]{mdframed}

% NOTE Definizione nuovo ambiente per evidenziare il codice
\makeatletter
	\newenvironment{btHighlight}[1][]
		{\begingroup\tikzset{bt@Highlight@par/.style={#1}}\begin{lrbox}{\@tempboxa}}
		{\end{lrbox}\bt@HL@box[bt@Highlight@par]{\@tempboxa}\endgroup}

	\newcommand\btHL[1][]{%
		\begin{btHighlight}[#1]\bgroup\aftergroup\bt@HL@endenv%
	}
	\def\bt@HL@endenv{%
		\end{btHighlight}%
		\egroup%
	}
	\newcommand{\bt@HL@box}[2][]{%
		\tikz[#1]{%
			\pgfpathrectangle{\pgfpoint{1pt}{0pt}}{\pgfpoint{\wd #2}{\ht #2}}%
			\pgfusepath{use as bounding box}%
			\node[anchor=base west, fill=orange!30,outer sep=0pt,inner xsep=1pt, inner ysep=0pt, rounded corners=3pt, minimum height=\ht\strutbox+1pt,#1]{\raisebox{1pt}{\strut}\strut\usebox{#2}};
		}%
	}
\makeatother

\newcommand{\greenHL}{\btHL[fill=green!30]}
\newcommand{\blueHL}{\btHL[fill=blue!30]}
\newcommand{\azureHL}{\btHL[fill=SkyBlue]}
\newcommand{\redHL}{\btHL}
\newcommand{\yellowHL}{\btHL[fill=yellow!60]}
\newcommand{\orangeHL}{\btHL[fill=orange!60]}

% NOTE: "listingsutf8" - permette di evitare problemi di codifica NEI FILE CARICATI ESTERNAMENTE
\usepackage{listingsutf8}
\usepackage{alltt}

% NOTE: Abbreviazioni inserimento codice
\newcommand\code[1]{\texttt{#1}}
\newcommand\html[1]{\lstinline[style = HTML]|#1|}
\newcommand\cc[1]{\lstinline[style = C]|#1|}
\newcommand\cpp[1]{\lstinline[style = [11]C++]|#1|}
\newcommand\gradle[1]{\lstinline[style = Gradle]|#1|}
\newcommand\java[1]{\lstinline[style = Java]|#1|}
\newcommand\javascript[1]{\lstinline[style = Javascript]|#1|}
\newcommand\jsp[1]{\lstinline[style = JSP]|#1|}
\newcommand\sml[1]{\lstinline[style = SML]|#1|}
\newcommand\sql[1]{\lstinline[style = SQL]|#1|}
\newcommand\xml[1]{\lstinline[style = XML]|#1|}

% NOTE: Rinomina la didascali
% Listing -> Codice
\renewcommand{\lstlistingname}{Codice}

% NOTE: Rinomina l'indice
% List of Listings -> Lista dei Codici
\renewcommand{\lstlistlistingname}{Lista dei Codici}

\lstloadlanguages{
	[11]C++,
	Java,
	HTML,
	ML,
	SQL,
	XML,
}

\lstset{
	inputencoding = utf8/latin1,	% codifica UTF-8
	extendedchars = true,			% allows extended characters in listings, that means (national) characters of codes 128–255.
	upquote = true,					% determines whether the left and right quote are printed ‘’ or `'. WARNING: This key  requires the textcomp package if true.
	texcl,							% Permette di mostrare formule matematiche nei commenti
	basicstyle = \footnotesize\ttfamily,
	numbers = none,					% Non mostra i numeri lateralmente
	breakatwhitespace = false,		% Non permette di spezzare il codice dove c'è uno spazio
	breaklines = true,				% Permette di spezzare il codice
	columns = fixed,				% incolonnamento corretto dei caratteri
	frame = H,						% Posizionamento corretto dal codice
	prebreak = \raisebox{0ex}[0ex][0ex]{\ensuremath{\rhookswarrow}},
	postbreak = \raisebox{0ex}[0ex][0ex]{\ensuremath{\rcurvearrowse}\space},
	escapeinside = {(*@}{@*)},		% %*codice latex*) CHANGED: (* *) -> (*@ "codice latex" @*)
	tabsize = 4,					% imposta la larghezza del tab pari a 4 spazi
	showspaces = false,				% non mostra gli spazi come caratteri
	showstringspaces = false,		% non mostra gli spazi nelle stringhe come caratteri
	showtabs = false,				% non mostra i tab come carattere
	frame = lines,					% aggiunge una linea sopra ed una linea sotto
	captionpos = b,					% Imposta la posizione della didascalia sul fondo
	aboveskip = 3mm,				% spazio superiore di 0.3 cm
	belowskip = 3mm,				% spazio inferiore di 0.3 cm
	% By default, listings does not support multi-byte encoding for source code. The extendedchar option only works for 8-bits encodings such as latin1. To handle UTF-8, you should tell listings how to interpret the special characters by defining them like so
	literate =
		{á}{{\'a}}1 {é}{{\'e}}1 {í}{{\'i}}1 {ó}{{\'o}}1 {ú}{{\'u}}1
		{Á}{{\'A}}1 {É}{{\'E}}1 {Í}{{\'I}}1 {Ó}{{\'O}}1 {Ú}{{\'U}}1
		{à}{{\`a}}1 {è}{{\`e}}1 {ì}{{\`i}}1 {ò}{{\`o}}1 {ù}{{\`u}}1
		{À}{{\`A}}1 {È}{{\'E}}1 {Ì}{{\`I}}1 {Ò}{{\`O}}1 {Ù}{{\`U}}1
		{ä}{{\"a}}1 {ë}{{\"e}}1 {ï}{{\"i}}1 {ö}{{\"o}}1 {ü}{{\"u}}1
		{Ä}{{\"A}}1 {Ë}{{\"E}}1 {Ï}{{\"I}}1 {Ö}{{\"O}}1 {Ü}{{\"U}}1
		{â}{{\^a}}1 {ê}{{\^e}}1 {î}{{\^i}}1 {ô}{{\^o}}1 {û}{{\^u}}1
		{Â}{{\^A}}1 {Ê}{{\^E}}1 {Î}{{\^I}}1 {Ô}{{\^O}}1 {Û}{{\^U}}1
		{œ}{{\oe}}1 {Œ}{{\OE}}1 {æ}{{\ae}}1 {Æ}{{\AE}}1 {ß}{{\ss}}1
		{ű}{{\H{u}}}1 {Ű}{{\H{U}}}1 {ő}{{\H{o}}}1 {Ő}{{\H{O}}}1
		{ç}{{\c c}}1 {Ç}{{\c C}}1 {ø}{{\o}}1 {å}{{\r a}}1 {Å}{{\r A}}1
		{€}{{\euro}}1 {£}{{\pounds}}1 {«}{{\guillemotleft}}1
		{»}{{\guillemotright}}1 {ñ}{{\~n}}1 {Ñ}{{\~N}}1 {¿}{{?`}}1
		% {$}{\$}1 % TODO: Da testare
	% Another possibility is to replace \usepackage{listings} (in the preamble) with \usepackage{listingsutf8}, but this will only work for \lstinputlisting{...}
}

% NOTE: Rinomina la didascalia dei codici
% NOTE: Rinomina l'indice dei codici
% Listing -> Codice
% List of Listings -> Lista dei Codici
\renewcommand{\lstlistingname}{Codice}
\renewcommand{\lstlistlistingname}{Lista dei Codici}

\lstdefinestyle{Gradle} {
	moredelim = *[s][\color{gray}]{'}{'},
	moredelim = *[s][\color{ForestGreen}]{"}{"},
	emphstyle = \bfseries,
	emph = {
		apply,
		% android,
		compileSdkVersion,
		buildToolsVersion,
		defaultConfig,
		minSdkVersion,
		targetSdkVersion,
		buildTypes,
		release,
		proguardFiles,
		getDefaultProguardFile,
		% dependencies,
		implementation,
		fileTree, testImplementation,
		androidTestImplementation
	},
	emphstyle = {[2]\bfseries\color{violet}},
	emph = {[2]
		applicationId,
		versionCode,
		versionName,
		testInstrumentationRunner,
		minifyEnabled
	},
	emphstyle = {[3]\bfseries\color{orange}},
	emph = {[3]
		true,
		false
	},
}

\lstdefinestyle{HTML} {
	language = HTML,
	morekeywords = {
		placeholder,
		pattern,
		autofocus,
		required,
	},
	moredelim = **[is][{\btHL[fill=green!30]}]{+++}{+++}
}

\lstdefinestyle{Java} {
	language = Java,
	commentstyle = \color{ashgrey},
	emphstyle = {\color{darkgray}},
	emphstyle = {[2]\bfseries},
	keywordstyle = \color{black}\bfseries,
	morecomment = [l][\color{blue}]{//+},		% single line blue comments
	morecomment = [l][\color{red}]{//-},		% single line red comments
	morecomment = [s][\color{blue}]{/*+}{*/},	% multiple line blue comments
	morecomment = [s][\color{red}]{/*-}{*/},	% multiple line red comments
	moredelim = [s][\color{ashgrey}]{/**}{*/},		% multiple line gray comments
	moredelim = [is][\soutthick]{|}{|},						% cancellato
	moredelim = **[is][\color{blue}]{^}{^},					% in blue cumulativo
	moredelim = **[is][\color{ForestGreen}]{***}{***},		% in verde cumulativo
	moredelim = **[is][\color{red}]{~}{~},					% [Alt + 126] in rosso cumulativo
	moredelim = **[is][\btHL]{---}{---},					% rimosso cumulativo
	moredelim = **[is][{\btHL[fill=blue!60]}]{Ž}{Ž},		% [Alt + 0142]  evidenzia cumulativo
	moredelim = **[is][{\btHL[fill=SkyBlue]}]{•}{•},		% [Alt + 0149]  evidenzia cumulativo
	moredelim = **[is][{\btHL[fill=orange!60]}]{Š}{Š},		% [Alt + 0138]  evidenzia cumulativo
	moredelim = **[is][{\btHL[fill=yellow!60]}]{‡}{‡},		% [Alt + 0135]  modificato cumulativo
	moredelim = **[is][{\btHL[fill=ForestGreen]}]{†}{†},	% [Alt + 0134]  evidenzia cumulativo
	moredelim = **[is][{\btHL[fill=green!30]}]{+++}{+++},	%  aggiunto cumulativo
	emph = {
		@Override,
		@param
	},
	emph = {[2]
		Note
	},
	morekeywords = {
		% Java Standard
		PrintWriter, BufferedReader, Integer,
		% Added for Android highlighting
		Intent, LayoutInflater, Menu, MenuInflater, String, Bundle, ViewHolder,
		FragmentManager, FragmentTransaction,
		Object, Object...,
		ArrayList, List, Bitmap, DataSetObserver,
		View, ViewGroup,
		CharSequence, Parcelable, Serializable,
		TextView, ImageView,
		ImageButton, EditText, ListView, ContextMenu,
		LinearLayout, RelativeLayout, TableLayout, FrameLayout,
		IBinder, Runnable,
		% SQLite
		SQLiteDatabase, MySQLiteHelper, Cursor, ArrayAdapter, ContentValues,
		Person, Messenger,
		% Socket Programming
		Socket, ServerSocket,
		InetAddress, DataOutputStream, PrintStream, OutputStream, InputStreamReader, StringTokenizer, FileInputStream,
		% Java Servlet Objects
		HttpServletRequest, HttpServletResponse,
		DateFormat, Integer,
	}
}

\lstdefinestyle{Javascript} {
	language = Javascript,
	commentstyle = \color{ashgrey},
	keywordstyle = \color{black}\bfseries,
}

\lstdefinestyle{JSP} {
	language = JSP,
	commentstyle = \color{ashgrey},
	keywordstyle = \color{black}\bfseries,
	moredelim = [s][\color{ashgrey}]{/**}{*/},		% multiple line gray comments
	moredelim = **[is][\color{red}]{~}{~},			% in rosso cumulativo
}

\lstdefinestyle{Kotlin} {
	language = Kotlin,
	commentstyle = \color{gray}\ttfamily,
	emphstyle = {\color{OrangeRed}},
	identifierstyle = \color{black},
	keywordstyle = \color{NavyBlue}\bfseries,
	% keywordstyle = \color{BurntOrange}\bfseries,
	stringstyle = \color{ForestGreen}\ttfamily,
}

\definecolor{ashgrey}{rgb}{0.7, 0.75, 0.71}
\lstdefinestyle{SML} {
	language = SML,
	commentstyle = \color{ashgrey},
	keywordstyle = \color{black}\bfseries,
	morekeywords = {
		val, type, datatype,
		fn, rec, fun,
		let, local, in, end,
	},
	moredelim = **[is][{\btHL[fill=orange!60]}]{Š}{Š},
	moredelim = **[is][{\btHL[fill=yellow!60]}]{‡}{‡},
	moredelim = **[is][{\btHL[fill=ForestGreen]}]{†}{†},
	moredelim = **[is][{\btHL[fill=green!30]}]{+++}{+++},
	moredelim = **[is][\btHL]{---}{---},
}

\lstdefinestyle{SQL} {
	language = SQL
}

\definecolor{ashgrey}{rgb}{0.7, 0.75, 0.71}
\definecolor{darkblue}{rgb}{0.0,0.0,0.6}
\lstdefinestyle{XML} {
	language = XML,
	commentstyle = \color{ashgrey},
	tag = **[s][\color{darkblue}\renewcommand\delimstyle{\color{black}}]<>,
	identifierstyle = \color{darkblue},
	keywordstyle = \color{cyan},
	stringstyle = \color{ForestGreen},
	moredelim = [is][\soutthick]{|}{|},
	% moredelim = [s][\color{ashgrey}]{<!--}{-->},
	moredelim = **[is][\color{red}]{~}{~},
	moredelim = **[is][\btHL]{---}{---},
	moredelim = **[is][{\btHL[fill=green!30]}]{+++}{+++},
	morekeywords = {
		android, application, encoding,
		id, intent-filter,
		layout_width, layout_height,
		name, orientation, text, type, title, xmlns, version,
	}
}

\makeatletter
\def\lst@DelimPrint#1#2{%
	#1%
		\begingroup
			\lst@mode\lst@nomode \lst@modetrue
			#2\delimstyle\lst@XPrintToken%
		\endgroup
		\lst@ResetToken
	\fi}
\makeatother
\newcommand\delimstyle{}

\lstdefinelanguage{XML} {
	% language = XML,
	morestring = [s]{"}{"},
	morestring = [s]{>}{<},
	morecomment = [l]{\#},
	% morecomment = [s]{<?}{?>},
	morecomment = [s]{<!--}{-->},
}

\lstdefinelanguage{Javascript} {
	language = HTML,
}

\lstdefinelanguage{JSP} {
	language = Java,
	alsolanguage = html,
}

\lstdefinelanguage{Kotlin} {
	sensitive = true,
	comment = [l]{//},
	morecomment = [s]{/*}{*/},
	morestring = [b]",
	morestring = [s]{"""*}{*"""},
	emph = {
		println, return@, forEach,
	},
	keywords = {
		package, as, typealias, this, super, val, var, fun, for, null,
		true, false, is, in, throw, return, break, continue, object, if,
		try, else, while, do, when, yield, typeof, yield, typeof, class,
		interface, enum, object, override, public, private, get, set, import, abstract,
	},
	keywords = {[2]
		@Deprecated, Iterable, Int, Integer, Float,
		Double, String, Runnable, dynamic,
	},
}

\lstdefinelanguage{SML} {
	language = ML,
}


% NOTE 'enumitem' - permette di personalizzare gli elenchi puntati
% NOTE lista in linea
\usepackage[inline]{enumitem}

% NOTE imposta il trattino negli elenchi puntati come marcatore
\renewcommand{\labelitemi}{\normalfont\bfseries\textendash}

% NOTE diminuisce GLOBALMENTE la distanza fra i punti
\setlist{itemsep = 3pt, topsep = 3pt}

% NOTE definizione liste personalizzate
% \newlist{<name>}{<type>}{<max-depth>}

\newlist{compactlist}{itemize}{2}
\setlist[compactlist]{label=\normalfont\bfseries\textendash,noitemsep, topsep = 0pt, parsep = 0pt, partopsep = 0pt}

\newlist{semicompactlist}{itemize}{2}
\setlist[semicompactlist]{label=\normalfont\bfseries\textendash,noitemsep, topsep = 0pt, parsep = 5pt, partopsep = 0pt}

% NB dipendenza 'set-tikz-macros'
\newlist{circledlist}{enumerate}{10}
\setlist[circledlist]{label=\circled{\arabic*}}

% IDEA: annotazioni - note a margine
% TODO: usalo per i commenti nella tesi
\usepackage{
	% todonotes,
	marginnote,
	mparhack,
	marginfix,
}

% WARNING: Mantenee l'ordine dei pacchetti è fonadamentale per non rompere la build
% WARNING: "semantic" da caricare sopo "amsmath"
% IDEA: math suymbols
% NOTE: "amsthm" - teoremi e dimostrazioni
% NOTE: "amsfonts" - nomi insiemi numerici
% NOTE: "amssymb" - leqslant & geqslant
% NOTE: "mathtools" - mathtools = amsmath + other stuff
% NOTE: "MnSymbol" - fornisce le freccie che utilizzo per andare a capo riga nelle liste di codice
% NOTE: "abraces" - angle brakets fine tuning
% NOTE: "braket" - permette l'uso di parentesi angolari
% NOTE: "nicefrac" - divisione in linea
% NOTE: "textgreek" - caratteri greci
% NOTE: "siunitx" - unità di misura del SI
\usepackage{
	,amsthm
	,amsfonts
	,amssymb
	,MnSymbol
}

% NOTE: "leqno" places the numbering of formulas on the left hand side instead of the right;
% NOTE: "fleqn" typesets displayed formulas left-aligned instead of centered;
\usepackage[
	,leqno
	% ,fleqn
]{mathtools}

\usepackage{
	% ,braket
	,semantic
	,empheq
	,nicefrac
}

\usepackage{witharrows}
\colorlet{brighter}{blue!50!white}
\WithArrowsOptions{tikz=brighter, MoreColumns}

% NOTE: comando definito dal package "amsmath"
\DeclarePairedDelimiter\norm{\lVert}{\rVert}
\DeclarePairedDelimiter\abs{\lvert}{\rvert}
\DeclarePairedDelimiter\Abs{\bigg\lvert}{\bigg\rvert}
\DeclarePairedDelimiter\Bracket{\lbrack}{\rbrack}

% NOTE: colorare simboli matematici
% tex.stackexchange.com/questions/21598/
\makeatletter
\def\mathcolor#1#{\@mathcolor{#1}}
\def\@mathcolor#1#2#3{%
	\protect\leavevmode
	\begingroup
		\color#1{#2}#3%
	\endgroup
}
\makeatother

% NOTE: Linguaggi formali e compilatori
\newcommand\produce{\longrightarrow}
\newcommand\deriva{\Rightarrow}

\usepackage{stackrel}

% TODO: simbolo spostato sulla destra
\newcommand\derivamultiplo{\stackrel{\ensuremath{+}}{\deriva}}
\newcommand\derivanumero[1]{\stackrel{\ensuremath{#1}}{\deriva}}

% NOTE: analisi
\renewcommand\restriction{\mathord{\upharpoonright}}

% NOTE: "uq" sta per "upquote"
\newcommand\uq{\ensuremath \text{\textquotesingle}}
\newcommand\bs[1]{\boldsymbol{#1}}

% NOTE: abbreviazioni
\newcommand\ob{\overbrace}
\newcommand\ub{\underbrace}
\newcommand\ubk{\underbracket}
\newcommand\us{\underset}
\newcommand\ul{\underline}
\newcommand\tn{\tn}
\newcommand\fns{\footnotesize}

% NB: modifico il comportamento del comando 'underbrace'
% NOTE: tex.stackexchange.com/questions/256894/
\makeatletter
\let\ams@underbrace=\underbrace
\def\underbrace#1_#2{%
	\setbox0=\hbox{$\displaystyle#1$}%
	\ams@underbrace{#1}_{\parbox[t]{\the\wd0}{\scriptsize #2}}%
}
\makeatother

% NOTE: simbolo di fine dimostrazione
\renewcommand\qedsymbol{\( \blacksquare \)}

% NOTE: teoremi, lemmi e nota bene
\newtheorem{theorem}{Teorema}
\newtheorem{lemma}[theorem]{Lemma}
\newtheorem{definition}{Definizione}[section]
\newtheorem{corollario}{Corollario}
\newtheorem{example}{Esempio}
\newtheorem{proposition}{Proposizione}

\newtheorem*{remark}{Ricorda}
\newtheorem*{note}{Nota}
\newtheorem*{observation}{Osservazione}
\newtheorem*{hint}{Suggerimento}

% - :bookmark: [Big O and related notations](texblog.org/2014/06/24/)
\newcommand\Omicron{\mathcal{O}}

% IDEA 'estione pagine
% NOTE 'pdfpages' - inserimento di pdf all'interno del documento
% NOTE 'afterpage' -  TODO
% NOTE 'lastpage' - riferimento all'ultima pagina (LastPage)
\usepackage{
	,pdfpages
	,afterpage
}

% NOTE Permette di inserisce una pagina bianca
\newcommand{\blankpage}{%
	\null%
	\thispagestyle{empty}%
	\addtocounter{page}{-1}%
	\newpage%
}

% NOTE 'fancyhdr' - permette la personalizzazione di testatina e piè di pagina
% WARNING non compatibile con la classe 'exam'
\usepackage{fancyhdr, lastpage}

% WARNING Commenta quando usi la documentclass 'exam'
% NOTE \pagestyle{footer}
\fancypagestyle{footer}{
	\fancyhf{}
	% \lfoot{\authorName}
	\cfoot{\footnotesize Pagina~\thepage\ di~\pageref{LastPage}}
	% \rfoot{\today}
	\renewcommand{\headrulewidth}{0pt}
	\renewcommand{\footrulewidth}{0pt}
}

% NOTE: "fancyhdr" - permette la personalizzazione di testatina e piè di pagina
% WARNING: non compatibile con la classe "exam"
\usepackage{fancyhdr}

% WARNING: Commenta quando usi la documentclass "exam"
\fancypagestyle{footer}{
	\fancyhf{}
	% \lfoot{\authorName}
	\cfoot{\footnotesize Pagina~\thepage\ di~\pageref{LastPage}}
	% \rfoot{\today}
	\renewcommand{\headrulewidth}{0pt}
	\renewcommand{\footrulewidth}{0pt} % default: 0.8pt
}

% NOTE toglie l'intentazione a tutto il documento
\setlength{\parindent}{0ex}
\setlength{\parskip}{1ex}
% \usepackage[onehalfspacing]{setspace}
\usepackage[parfill]{parskip}

% NOTE 'csquotes' - permette di personalizzare le citazioni
\usepackage{csquotes}

% NOTE: Ridefinisce le virgolette
\DeclareQuoteStyle{italian}%
	{\textquotedblleft}
	[\textquotedblleft]
	{\textquotedblright}
		[0.05em]
	{\textquoteleft}
	[\textquoteleft]
	{\textquoteright}

% NOTE: Imposta il tipo di virgolette
\setquotestyle{italian}

% TODO: mi dervirà in futuro
% \usepackage[
%     ,left = \flqq{}
%     ,right = \frqq{}
%     ,leftsub = \flq{}
%     ,rightsub = \frq{}
% ]{dirtytalk}

% NOTE formatta una quotazione all'inizio di una sezione
\usepackage{epigraph}

% \epigraphsize{\small}% Default
\setlength\epigraphwidth{8cm}
\setlength\epigraphrule{0pt}

\usepackage{etoolbox}

\makeatletter
\patchcmd{\epigraph}{\@epitext{#1}}{\itshape\@epitext{#1}}{}{}
\makeatother

% IDEA: rotazione
% NOTE: "pdfpages" - inserimento di pdf all'interno del documento
% NOTE: "rotating" - rotazione tabelle
\usepackage{
	pdflscape,
	rotating,
}

% IDEA tabelle
% NOTE 'array' - permette di creare delle colonne personalizzate
% NOTE 'bigstrut' -
% NOTE 'booktabs' - genera filetti professionali per le tabelle
% NOTE 'colortbl' - righe e celle colorate
% NOTE 'diagbox' - diagonal rule on a cell
% NOTE 'ltablex' - crea tabelle dalla larghezza dinamica su più pagine
% NOTE 'makecell' - supporta layout comuni per le intestazioni, può creare celle multi-allineate
% NOTE 'multirow' - tabelle con righe multilinea
% NOTE 'tabularx' - crea tabelle dalla larghezza dinamica
\usepackage{
	,array
	,tabu
	,bigstrut
	,booktabs
	% ,xcolor NOTE just a dependency
	,colortbl
	,diagbox
	,float % floatrow
	% ,ltablex % WARNING conflitto con 'tabularx'
	,makecell
	,multirow
	,tabularx
}

% NOTE linea sottile grigia all'interno della tabella
\setlength{\lightrulewidth}{0.1pt}
\newcommand{\lightrule}{%
	\arrayrulecolor{black!30}%
	\midrule[\lightrulewidth]%
	\arrayrulecolor{black}}

% NOTE dipendenza package 'array'
% OPTIMIZE rimuovere questo codice quando lo trovi
% \newcolumntype{!}{>{\global\let\currentrowstyle\relax}}
% \newcolumntype{^}{>{\currentrowstyle}}
% \newcommand{\rowstyle}[1]{%
% 	\gdef\currentrowstyle{#1}%
% 	#1\ignorespaces}

% NOTE: optional
% tex.stackexchange.com/questions/177164/
% tex.stackexchange.com/questions/341656/
\usetikzlibrary[external]
% \usepackage{pgfplots}
% \usepgfplotslibrary[
% 	external
% ]

% Macro holding the externalized sub-directory
\newcommand{\externaldirectory}{_tikz-cache/}

% NOTE: opzioni per la pre-compilazioni delle immagini create con tikz
\tikzexternalize[
	mode = graphics if exists,
	% figure list = true,
	% All externalized graphics go to the \externaldirectory
	prefix = \externaldirectory
]
% Externalise only on-demand.
\tikzexternaldisable

% IDEA: DISEGNARE GRAFICI
% NOTE: "tikz" - pacchetto completo per disegnare su LaTeX
\usepackage{tikz}

% NOTE: sotto-librerie del pacchetto tikz
\usetikzlibrary{
	arrows,
	calc,
	intersections,
	% matrix,
	positioning,
	% shapes.geometric,
	tikzmark,
	% decorations.text,
	% decorations.pathmorphing
}

% NOTE: Simbolo per riferimento a materiali esterni
\newcommand{\ExternalLink}{
	\tikz[x = 1.2ex, y = 1.2ex, baseline = -0.05ex]{
		\begin{scope}[x = 1ex, y = 1ex]
			\clip (-0.1,-0.1) --++ (-0, 1.2) --++ (0.6, 0) --++ (0, -0.6) --++ (0.6, 0) --++ (0, -1);
			\path[draw, line width = 0.5, rounded corners = 0.5] (0,0) rectangle (1,1);
		\end{scope}
		\path[draw, line width = 0.5] (0.5, 0.5) -- (1, 1);
		\path[draw, line width = 0.5] (0.6, 1) -- (1, 1) -- (1, 0.6);
	}
}

% https://latex.org/forum/viewtopic.php?t=22367
\DeclareRobustCommand\circled[1]{%
	\tikz[baseline = (char.base)]{%
		\node[draw, circle, inner sep = 1pt]%
			(char) {#1};%
	}
}

\input{more-graph}

% NOTE: "tocloft" - gestione dei Table Of Content
% NB: "titles" - evita errori di caricamento dei pacchetti
\usepackage[titles]{tocloft}

% HACK: Change it when you use "book" as document class
% \renewcommand{\cftchapafterpnum}{\vspace{10pt}}
\renewcommand{\cftsecafterpnum}{\vspace{10pt}}

% \let\Chapter\chapter
% \def\chapter{\addtocontents{lol}{\protect\addvspace{10pt}}\Chapter}
\let\Section\section%
\def\section{\addtocontents{lol}{\protect\addvspace{10pt}}\Section}

% NOTE: La numerazione delle sezione si azzera all'inizio di una nuova parte
\counterwithin*{section}{part}

% IDEA: definizione nuovi comandi o ambienti
% NOTE: "chngcntr" - defines the "\counterwithin" command, useful with parts of a document
% NOTE: "comment" - fornisce l'ambiente dei commenti
% NOTE: "etoolbox" - contiene il costrutto if-then-else più altri strumenti utili
% NOTE: "pgffor" - fornisce il costrutto "foreach"
% NOTE: "textcomp" - definisce la macro "textquotesingle" e formatta i numeri
% NOTE: "xargs" - use more than one optional parameter in a new commands
\usepackage{
	chngcntr,
	comment,
	etoolbox,
	pgffor,
	xargs,
	xstring,
	calc
}


% NOTE dipendenza 'xcolor'
% \colorlet{darker}{blue!50!black}

% WARNING it produces a clashing error with the tocloft package if the TOC is displayed
% NB bisogna caricarlo per ultimo
% NOTE gestisce i link testuali all'interno del documento
\usepackage{hyperref}
\hypersetup{
	,colorlinks	= true
	% ,allcolors = darker
	,linkcolor 	= red
	,anchorcolor = black
	,citecolor 	= green
	,filecolor 	= cyan
	,menucolor 	= red
	,runcolor 	= cyan
	,urlcolor 	= magenta
	,pdfauthor	= {Nardi, Emanuele}
	% pdfpagemode=FullScreen
}

% NOTE specifica la mail con link
\newcommand{\mail}[1]{\href{mailto:#1}{\texttt{#1}}}

% WARNING deve essere caricato dopo il pacchetto 'amsmath'
% NOTE non funziona come dovrebbe, leggere la documentazione
% \usepackage{cleveref}

% IDEA compilare indipendente i file
% NOTE "subfiles" - TODO
% NOTE "standalone" - TODO
% NOTE "docmute" - TODO
% WARNING: NON CAMBIARE L'ORDINE DI CARICAMENTO DEI PACCHETTI
\usepackage[
	,subpreambles	= true
	,mode			= buildnew
]{standalone}

\usepackage{subfiles}
% \usepackage{docmute}


% TODO: implementare in tutti i documenti
% % WARNING: può creare probemi a tempo di compilazione
\hypersetup{
	pdfinfo = {
		pdffitwindow = true,	 		% window fit to page when opened
		pdfnewwindow = true,			% links in new PDF window
		pdftitle = {\pdfTitle},			% PDF's title
		pdfsubject = {\subject},		% subject of the document
		pdfkeywords = {\tags},			% list of keywords
		pdfauthor = {\authorName},		% author of the document
		pdfcreator = {\authorName},		% creator of the document
		pdfproducer = {\authorName},	% producer of the document
	}
}

%
% NOTE: Definizione nuovi comandi per il pacchetto "notes"
\newcommandx{\insicuro}[2][1=]{\todo[linecolor=red,backgroundcolor=red!25,bordercolor=red,#1]{#2}}
\newcommandx{\cambia}[2][1=]{\todo[linecolor=blue,backgroundcolor=blue!25,bordercolor=blue,#1]{#2}}
\newcommandx{\info}[2][1=]{\todo[linecolor=OliveGreen,backgroundcolor=OliveGreen!25,bordercolor=OliveGreen,#1]{#2}}
\newcommandx{\miglioramento}[2][1=]{\todo[linecolor=Plum,backgroundcolor=Plum!25,bordercolor=Plum,#1]{#2}}
\newcommandx{\nascosto}[2][1=]{\todo[disable,#1]{#2}}

%

% NOTE: Permette di inserisce una pagina bianca
\newcommand{\blankpage}{%
	\null%
	\thispagestyle{empty}%
	\addtocounter{page}{-1}%
	\newpage%
}

% NOTE: Simbolo per riferimento a materiali esterni
\newcommand{\ExternalLink}{
	\tikz[x = 1.2ex, y = 1.2ex, baseline = -0.05ex]{
		\begin{scope}[x = 1ex, y = 1ex]
			\clip (-0.1,-0.1) --++ (-0, 1.2) --++ (0.6, 0) --++ (0, -0.6) --++ (0.6, 0) --++ (0, -1);
			\path[draw, line width = 0.5, rounded corners = 0.5] (0,0) rectangle (1,1);
		\end{scope}
		\path[draw, line width = 0.5] (0.5, 0.5) -- (1, 1);
		\path[draw, line width = 0.5] (0.6, 1) -- (1, 1) -- (1, 0.6);
	}
}

\newcommand{\mail}[1]{\href{mailto:#1}{\texttt{#1}\ExternalLink}}

\newcommand{\mysection}[2]{\section[#1]{#1\\[.5ex]\normalsize\textit{#2}}}
\newcommand{\mysubsection}[2]{\subsection[#1]{#1\\[.5ex]\normalsize\textit{#2}}}

\newcommand{\omissis}{[\textellipsis\unkern]}

% NOTE: Linea spessa sul testo sottostante
\newcommand{\soutthick}[1]{
	\renewcommand{\ULthickness}{1.0pt}	% 2.4
		\sout{#1}%
	\renewcommand{\ULthickness}{.4pt}	% Resetting to ulem default
}

% ---------------------------------------------------------------------------- %
% ---------------------------------------------------------------------------- %
% ---------------------------------------------------------------------------- %
