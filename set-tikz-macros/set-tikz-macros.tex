% NOTE: Simbolo per riferimento a materiali esterni
\newcommand{\ExternalLink}{
	\tikz[x = 1.2ex, y = 1.2ex, baseline = -0.05ex]{
		\begin{scope}[x = 1ex, y = 1ex]
			\clip (-0.1,-0.1) --++ (-0, 1.2) --++ (0.6, 0) --++ (0, -0.6) --++ (0.6, 0) --++ (0, -1);
			\path[draw, line width = 0.5, rounded corners = 0.5] (0,0) rectangle (1,1);
		\end{scope}
		\path[draw, line width = 0.5] (0.5, 0.5) -- (1, 1);
		\path[draw, line width = 0.5] (0.6, 1) -- (1, 1) -- (1, 0.6);
	}
}

% https://latex.org/forum/viewtopic.php?t=22367
\DeclareRobustCommand\circled[1]{%
	\tikz[baseline = (char.base)]{%
		\node[draw, circle, inner sep = 1pt]%
			(char) {#1};%
	}
}

% NOTE: tex.stackexchange.com/questions/51019/
\renewcommand{\tikzmark}[1]{\tikz[overlay,remember picture] \node (#1) {};}
\newcommand*{\AddNote}[6]{%
    \begin{tikzpicture}[overlay, remember picture]
        \draw [decoration={brace,amplitude=0.5em}, decorate, thick, #5]
			($(#3)!(#2.south)!($(#3)-(0,1)$)$) --
            ($(#3)!(#1.north)!($(#3)-(0,1)$)$)
                node [align=left, text width=#4, pos=0.5, anchor=east] {#6};
    \end{tikzpicture}
}
